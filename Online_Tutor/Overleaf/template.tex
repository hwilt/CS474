%%%%%%%%%%%%%%%%%%%%%%%%%%%%%%%%%%%%%%%%%
% Journal Article
% LaTeX Template
% Version 1.4 (15/5/16)
%
% This template has been downloaded from:
% http://www.LaTeXTemplates.com
%
% Original author:
% Frits Wenneker (http://www.howtotex.com) with extensive modifications by
% Vel (vel@LaTeXTemplates.com)
%
% License:
% CC BY-NC-SA 3.0 (http://creativecommons.org/licenses/by-nc-sa/3.0/)
%
%%%%%%%%%%%%%%%%%%%%%%%%%%%%%%%%%%%%%%%%%

%----------------------------------------------------------------------------------------
%	PACKAGES AND OTHER DOCUMENT CONFIGURATIONS
%----------------------------------------------------------------------------------------

\documentclass[twoside,twocolumn]{article}

\usepackage{blindtext} % Package to generate dummy text throughout this template 

\usepackage[sc]{mathpazo} % Use the Palatino font
\usepackage[T1]{fontenc} % Use 8-bit encoding that has 256 glyphs
\linespread{1.05} % Line spacing - Palatino needs more space between lines
\usepackage{microtype} % Slightly tweak font spacing for aesthetics

\usepackage[english]{babel} % Language hyphenation and typographical rules

\usepackage[hmarginratio=1:1,top=32mm,columnsep=20pt]{geometry} % Document margins
\usepackage[hang, small,labelfont=bf,up,textfont=it,up]{caption} % Custom captions under/above floats in tables or figures
\usepackage{booktabs} % Horizontal rules in tables

\usepackage{lettrine} % The lettrine is the first enlarged letter at the beginning of the text

\usepackage{enumitem} % Customized lists
\setlist[itemize]{noitemsep} % Make itemize lists more compact

\usepackage{abstract} % Allows abstract customization
\renewcommand{\abstractnamefont}{\normalfont\bfseries} % Set the "Abstract" text to bold
\renewcommand{\abstracttextfont}{\normalfont\small\itshape} % Set the abstract itself to small italic text

\usepackage{titlesec} % Allows customization of titles
\renewcommand\thesection{\Roman{section}} % Roman numerals for the sections
\renewcommand\thesubsection{\roman{subsection}} % roman numerals for subsections
\titleformat{\section}[block]{\large\scshape\centering}{\thesection.}{1em}{} % Change the look of the section titles
\titleformat{\subsection}[block]{\large}{\thesubsection.}{1em}{} % Change the look of the section titles

\usepackage{fancyhdr} % Headers and footers
\pagestyle{fancy} % All pages have headers and footers
\fancyhead{} % Blank out the default header
\fancyfoot{} % Blank out the default footer
\fancyhead[C]{Java Quiz $\bullet$ April 2022} % Custom header text
\fancyfoot[RO,LE]{\thepage} % Custom footer text

\usepackage{titling} % Customizing the title section

\usepackage{hyperref} % For hyperlinks in the PDF

\usepackage{biblatex} %Imports biblatex package
\usepackage{csquotes}
\addbibresource{wilt.bib} %Import the bibliography file

%----------------------------------------------------------------------------------------
%	TITLE SECTION
%----------------------------------------------------------------------------------------

\setlength{\droptitle}{-4\baselineskip} % Move the title up

%\pretitle{\begin{center}\Huge\bfseries} % Article title formatting
%\posttitle{\end{center}} % Article title closing formatting
\title{Online Java Tutor - Quiz} % Article title
\author{%
\textsc{Henry Wilt} \\[1ex]%\thanks{A thank you or further information}  % Your name
\normalsize Ursinus College \\ % Your institution
\normalsize \href{mailto:hewilt@ursinus.edu}{hewilt@ursinus.edu} % Your email address
%\and % Uncomment if 2 authors are required, duplicate these 4 lines if more
%\textsc{Jane Smith}\thanks{Corresponding author} \\[1ex] % Second author's name
%\normalsize University of Utah \\ % Second author's institution
%\normalsize \href{mailto:jane@smith.com}{jane@smith.com} % Second author's email address
}
\date{April 6, 2022} % Leave empty to omit a date
\renewcommand{\maketitlehookd}{%
\begin{abstract}
\noindent The goal for this assignment was to write an online tutor\cite{wmonganweb}. The tutor I created was for beginner Java Students. This assignment is for my Human Computer Interaction course and is supposed to teach us how to use psychological triggers to develop the tutor. % Dummy abstract text - replace \blindtext with your abstract text
\end{abstract}
}

%----------------------------------------------------------------------------------------

\begin{document}

% Print the title
\maketitle

%----------------------------------------------------------------------------------------
%	ARTICLE CONTENTS
%----------------------------------------------------------------------------------------

\section{About the Tutor}

\lettrine[nindent=0em,lines=2]{I}t started off with picking a topic that I wanted to do for a tutorial for beginning java students. I started off with thinking what should the tutorial should include, I thought first of classes, objects, variables, and methods\cite{assignmentgithub}. This is a very simple tutor that can be expanded with more learning tutorials like Object Oriented Design, Polymorphism, and more Data Structures like arrays and hashmaps. This can easily added in with more questions to the quiz at the moment or different quizzes that can be added in other methods. The quiz in general is using two different dictionaries, one for the questions, and one for the answers for each question. This is then looped through printing the questions and asking the user for input to get the answer. At the end of the quiz, the student will get a score that will printed out for the student to see if they are getting better or worse. This can easily be added into the student object which then you can see how each student is doing in the class or tutor group. The code is expandable to the add more tutorials, quizzes, and more students.


%------------------------------------------------

\section{Conclusion}
In conclusion, I thought this was a good assignment. This didn't take me that long to do, about 4-6 hours over a couple days. This assignment has so many possibilities to do for the future of the project.  



%----------------------------------------------------------------------------------------
%	REFERENCE LIST
%----------------------------------------------------------------------------------------

\printbibliography %Prints bibliography

%----------------------------------------------------------------------------------------

\end{document}
